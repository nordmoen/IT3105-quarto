\section{Implementation}\label{implementation}
Since we did the last project in Java creating a highly parallel algorithm for
roll-out simulation we wanted a different challenge this time around. Because
of this we decided that we wanted to create our implementation in Python, but
because Python can be quite slow compared to other languages we wanted to implement the
time critical Minimax algorithm in C and interface that with our Python code.

Since our implementation had to be supported in both Python and in C we decided
quite early on that we needed to represent pieces as simple as possible. After
some testing we found that representing each piece as a byte worked very well.
Using a byte where each bit represented an attribute gave us a very easy task
of sending information between Python and C and made our implementation very
fast. There were several advantages representing our pieces as a single byte,
not only did it help us interface Python and C, but the representation would
not have to change if we needed to play with someone else using this same
representation. This is because the method of comparing similarity would not
have to change even though we gave different attributes to different bits in
the implementation.

Not everything could interface as easily as our pieces though. Since Python is 
inherently object-oriented our representation of the \quarto{} board was also 
implemented as an object, this did provide us with some headaches which we 
had to overcome.
In the end we managed to get everything working and we got quite good speed 
out of it which is reflected in our results.

The advantage of doing the project this ways is that we could easily cooperate
with other groups playing in Python. And the ease of working in Python is
quite different than working in pure C.

As mentioned before we use a byte\footnote{In the code we have used integers and
unsigned char to represent it in code} to represent the individual pieces and to
compare equality. All one has to do is "AND" those bytes together to see if all
pieces share an attribute. The problem with that approach is that we only capture
available attributes not an absent of attributes, e.g. if we have 1100 AND 1010
we only detect that they share the first attribute not the last. To overcome this
we took each piece and applied "XOR" against 1111 to produce the compliment.
This would mean that the two pieces above also have their compliments compared
i.e. 1100 XOR 1111 == 0011 and 1010 XOR 1111 == 0101, 0011 AND 0101 which now will correctly
classify the last attribute as shared.
