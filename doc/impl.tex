\section{Implementation}\label{implementation}
Since we did the last project in Java creating a highly parallel algorithm for
roll-out simulation we wanted a different challenge this time around. Because
of this we decided that we wanted to create our implementation in Python, but
because Python can be quite slow compared to others we wanted to implement the
time critical Minimax algorithm in C and interface that with our Python code.

Since our implementation had to be supported in both Python and in C we decided
quite early on that we needed to represent pieces as simple as possible. After
some testing we found that representing each piece as an byte worked very well.
Using a byte where each bit represented an attribute gave us a very easy task
of sending information between Python and C and made our implementation very
fast. There were several advantages representing our pieces as a single byte,
not only did it help us interface Python and C, but the representation would
not have to change if we needed to play with someone else using this same
representation. This is because the method of comparing similarity would not
have to change even though we gave different attributes to different bits in
the implementation.

Not everything could interface as easily as our pieces though. Since Python is inherently
object-oriented our representation of the \quarto{} board was also implemented as
an object, but this did provide us with some headaches which we had to overcome.
In the end we managed to get everything working and we got quite good speed out of
it which is reflected in our results.
