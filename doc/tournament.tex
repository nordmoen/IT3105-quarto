\section{Tournament}\label{tournament}
When coding for the tournament we did not face many additional challenges, 
since our internal representation of a \quarto{} piece as a byte was 
consistent with the representation used by the tournament host. And although 
the mapping of attributes onto bits was chosen different by the players 
competing in the tournament, the comparison of pieces both internally and 
externally was never affected by this mapping. The mapping was merely used for 
displaying board-states visually.
What we had to was implement a class player.py that interfaces with our 
internal syntax used in minimax\_player.py to the syntax used by the 
tournament host. This was necessary due to differences in naming of functions 
and ordering of parameters.

Since we were a bit late to the tournament compared to the other players we did
not develop our player to the same specifications as the other players. This
meant that our implementations expected to receive our piece and board 
implementation and not pure integers as the other players had. This meant that
our player.py had to translate its arguments in order for it to be able
to pass on the correct arguments to our real minimax player.

All in all we did not have to do much of anything in order to participate in the
tournament. Cooperating with the other players went quite uneventfully and
everything worked very well in the end.
