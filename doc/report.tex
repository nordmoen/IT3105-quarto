\documentclass[titlepage, a4paper]{article}
\usepackage[colorlinks=true, linkcolor=black]{hyperref}
\usepackage{graphicx}
\usepackage{listings}
\usepackage[utf8]{inputenc}
\usepackage[english]{babel}

%Define a quarto command so we write quarto the same way everywhere
\newcommand{\quarto}{Quarto!}

\title{
	\quarto{} Minimax player \\
	IT3105 \\
}
\author{
	Nordmoen, Jørgen H. \\
	Østensen, Trond
}

\date{\today}

\begin{document}
\pagenumbering{RomanType}
\maketitle

\begin{abstract}\label{abstract}
	This paper is an introduction to our \quarto{} minimax player. In this we will try to
	explain how we created our implementation, what sort of decisions the player makes
	the reasoning behind it and our result both against other configurations of our player
	and against other students minimax implementations.
\end{abstract}

\newpage
\tableofcontents
\listoffigures

\pagenumbering{arabic}

\section{Introduction}\label{intro}
The game of \quarto{}\footnote{Explanation of rules in video form: 
\url{https://www.youtube.com/watch?feature=player_embedded&v=P6dy2eaYmos#!}}
is a quite simple game regarding its rules, but the hard part
comes into play when us humans try to remember all the different attributes of the
game. This is where a game playing algorithm comes into play. With an algorithm
memory no longer becomes an issue as the algorithm can enumerate, if possible, and search
through the whole instance space. In this assignment
we were tasked with creating a minimax\footnote{Explanation of minimax from Wikipedia:
\url{https://en.wikipedia.org/wiki/Minimax}} player with Alpha-beta pruning\footnote{
Further explanation on Wikipedia:\url{https://en.wikipedia.org/wiki/Alpha-beta_pruning}}.
The algorithm it self is not the most difficult one to implement, thats not to say that
we didn't need to debug our code, but the challenge is coming up with a good set of
heuristics to evaluate an intermediate state. We will first introduce our implementation
in a birds eye view, we will then explain our chosen heuristics in more detail before
we move on to our experience in the tournament. Lastly we will describe our results, both
against our own implementation and our tournament results.


\section{Implementation}\label{implementation}
Since we did the last project in Java creating a highly parallel algorithm for
roll-out simulation we wanted a different challenge this time around. Because
of this we decided that we wanted to create our implementation in Python, but
because Python can be quite slow compared to others we wanted to implement the
time critical Minimax algorithm in C and interface that with our Python code.

Since our implementation had to be supported in both Python and in C we decided
quite early on that we needed to represent pieces as simple as possible. After
some testing we found that representing each piece as an byte worked very well.
Using a byte where each bit represented an attribute gave us a very easy task
of sending information between Python and C and made our implementation very
fast. There were several advantages representing our pieces as a single byte,
not only did it help us interface Python and C, but the representation would
not have to change if we needed to play with someone else using this same
representation. This is because the method of comparing similarity would not
have to change even though we gave different attributes to different bits in
the implementation.

Not everything could interface as easily as our pieces though. Since Python is inherently
object-oriented our representation of the \quarto{} board was also implemented as
an object, but this did provide us with some headaches which we had to overcome.
In the end we managed to get everything working and we got quite good speed out of
it which is reflected in our results.


\section{Heuristics}\label{heuristics}
The state evaluation function implemented for the minimax algorithm searches 
through each end-state, looking for instances of board-states that are known  
to affect the outcome of the match.
Our heuristic function is set, arbitrarily, to return a value [-100, 100], 
where a higher number means a higher valued state.

We have implemented three distinct intermediate state evaluation heuristics
and two final state evaluators. The intermediate state evaluators try to assign
a value to a state telling us how good or bad that state is if we would end up
in that state. The final state evaluators evaluates the final board state telling
us if there is a winning player or if the players tied.

\subsection{Wins}\label{win}
\begin{figure}[htb]
\includegraphics{pictures/win.png}
\caption[A \quarto{} win]{A winning board state in \quarto{}}
\label{fig:win}
\end{figure}
The most obvious board-state to look for is a win(please see figure \ref{fig:win}), 
i.e. four pieces in a line  
sharing at least one attribute. To do this we compare all the pieces in each 
row, column and diagonal with each other and if a line is found where all 
pieces are equal we return a value of 100.

\subsection{Ties}
\begin{figure}[htb]
\includegraphics{pictures/tie.png}
\caption[A tie in \quarto{}]{A tied board state in \quarto{}}
\label{fig:tie}
\end{figure}
Another nice board-state to look out for is a tie(see figure \ref{fig:tie}), 
when all sixteen pieces have 
been placed on the board, but no win has been achieved. Checking this is merely 
counting if there are 16 pieces placed on the board, and if there is a  win. 
Since two optimal players will always tie\footnote{Proof of optimal play 
resulting in a tie: \url{http://web.archive.org/web/20041012023358/http://ssel.vub.ac.be/Members/LucGoossens/quarto/quartotext.htm}}, 
this is evaluated as 0, making sure that the minimax player will always 
choose a tie over a possible loss.

\subsection{Guaranteed losses}
\begin{figure}[htb]
\includegraphics{pictures/gloss.png}
\caption[A guaranteed loss in \quarto{}]{A board state showing a guaranteed loss in \quarto{}}
\label{fig:gloss}
\end{figure}
When there are two distinct 3-piece lines which have opposite values of at least 
one attribute(see figure \ref{fig:gloss}), 
giving any piece to the opponent is a  guaranteed loss.
When we search for these states, we start of by finding all 3-piece lines on the 
board, then compare them to see if any of them have opposite values of an attribute.
Since a loss is the worst possible outcome of a game of \quarto{}, these states 
are valued at -100 in order to make the minimax player steer away from them.

\subsection{3-piece lines}
\begin{figure}[htb]
\includegraphics{pictures/3-odd.png}
\caption[A 3-piece line in \quarto{}]{A board state showing a 3-piece line with 
an odd number of winning pieces}
\label{fig:3-odd}
\end{figure}

\begin{figure}[htb]
\includegraphics{pictures/3-even.png}
\caption[A 3-piece line in \quarto{}]{A board state showing a 3-piece line with 
an even number of winning pieces}
\label{fig:3-even}
\end{figure}
A board containing multiple 3-piece lines makes it difficult to give the 
opponent a non-winning piece to play. However, depending on whether there is 
an odd(see figure \ref{fig:3-odd}) or even(see figure \ref{fig:3-even}) 
number of pieces that can be placed in the final slot of a 
3-piece line, it is possible to force the opponent into returning a winning 
piece. With an even number of pieces the opponent can force us into giving
him a winning piece, and the other way around with and odd number of pieces. 
Since this is a risky strategy, the 3-piece lines with an even number 
of winning pieces are weighted heavier than the 3-piece lines with an odd 
number of winning pieces.

We have had a bit of debate around this last heuristic. Since it can be shown
that an optimal player always would tie the last heuristic can't force an opponent
into a losing situation, since the player could always finish off the line with
a non winning piece before being forced to give away a winning piece. For this
reason we were unsure whether or not the heuristics should be included or not.
In the end we included it because it should not hinder our performance, but
it can, against non optimal players, put us in a situation where we can win.

%TODO: include pictures


\section{Tournament}\label{tournament}
When coding for the tournament we did not face many additional challenges, 
since our internal representation of a \quarto{} piece as a byte was 
consistent with the representation used by the tournament host. And although 
the mapping of attributes onto bits was chosen different by the players 
competing in the tournament, the comparison of pieces both internally and 
externally was never affected by this mapping. The mapping was merely used for 
displaying board-states visually.
What we had to was implement a class player.py that interfaces with our 
internal syntax used in minimax\_player.py to the syntax used by the 
tournament host. This was necessary due to differences in naming of functions 
and ordering of parameters.

Since we were a bit late to the tournament compared to the other players we did
not develop our player to the same specifications as the other players. This
meant that our implementations expected to receive our piece and board 
implementation and not pure integers as the other players had. This meant that
our player.py had to translate its arguments in order for it to be able
to pass on the correct arguments to our real minimax player.

All in all we did not have to do much of anything in order to participate in the
tournament. Cooperating with the other players went quite uneventfully and
everything worked very well in the end.


\section{Results}\label{results}

\subsection{Local}\label{results:local}

\subsection{Tournament}\label{results:tournament}


%TODO: there needs to be more text here

%\section{Future work}\label{future}

\appendix
\lstset{language=bash, frame=single, breaklines=true}
\lstloadlanguages{bash}
\section{Run configuration}\label{run configuration}
We have included our parameters below for the different results that we got
in the result section(see section \ref{results}). This should enable anyone to clone
our git repository and get about the same results out.

\begin{lstlisting}[label=lst:novice vs random, caption=Novice compared to random
player with 500 games]
$ python main.py game --player1 random --player2 novice -r 500 -s
\end{lstlisting}

\begin{lstlisting}[label=lst:novice vs minimax, caption=Novice compared to
minimax player with 500 games]
$ python main.py game --player1 minimax 3 6 --player2 novice -r 500 -s
\end{lstlisting}

\begin{lstlisting}[label=lst:minimax3 vs minimax4, caption=Minimax with a
depth of 4 compared to a minimax player with a depth of 3]
$ python main.py game --player1 minimax 3 6 --player2 minimax 4 6 -r 500 -s
\end{lstlisting}

\lstinputlisting[label=lst:switch code, caption=
Code to test most minimax depth against each other]{graphs/test_minimax_depth.sh}


\end{document}
